\documentclass [a4paper]{article}
\usepackage[english]{babel}
\usepackage[
	pdftex,
	pdftitle={Multivio: Project description},
	pdfauthor={Miguel Moreira},
	pdfsubject={Multivio a new project.}, 
	pdfkeywords={PDF, Multivio},
	pdfdisplaydoctitle=true,
	colorlinks=true,
	urlcolor=blue
	]
{hyperref}
\usepackage{url}

\title{Multivio: Project description}
\author{Miguel Moreira}

\begin{document}

\maketitle

\section{Introduction}

Multivio is an Internet-based application for browsing and accessing digital
documents.

It was originally motivated by the requirements of the
\href{http://doc.rero.ch/}{digital library RERO DOC}. However, it is meant to
be open and adjustable in order to smoothly integrate with other document
servers. It is also modular and extensible in order to cope with the continuous
evolution of digital formats.

The application is built in-house by \href{http://www.rero.ch/}{RERO} (the
'''Library Network of Western Switzerland''') under an open-source license (to
be defined), and integrates existing open-source software components. Its
development is co-sponsored in the framework of the project '''E-lib.ch: Swiss
Electronic library''' (see below).

\section{Context}

In general, when users access the contents of digital libraries, they usually
dependent on particular software tools installed in their computer system,
suitable to each kind of content being accessed. These multiple tools offer a
scattered view of the available content since they consider the digital files
individually without taking into account logical relations between groups of
files belonging to specific titles or collections. This situation implies a
fracture between the phase when users search or navigate the object
descriptions and the moment when they access the objects themselves.

Indeed, the approach applied by most catalogues or document servers allowing
access to digital content is to simply provide a hypertext link allowing the
corresponding files to be downloaded. Then, on the user's side, a suitable tool
(that must be installed beforehand) allows the file to be viewed after
download. This approach is relatively satisfying in certain contexts,
particularly in the case of single volume monographs, for example in PDF
format, given that the large majority of Internet users possess PDF-viewing
software such as \href{http://www.adobe.com/products/reader/}{Adobe Reader} or alike
installed in his/her system. The same principle applies for other file types,
each relying on a specific and adequate software tool. But it has limitations
in the presence of documents of higher complex structure demanding navigation
capabilities across several files, such as serials, collections, multivolume
monographs and other aggregates. In addition, even in the case of plain,
single-file documents the simple download option is not always the most
appropriate, especially when the document has been found after a query operated
by the user in a search engine, in which case the queried expression should be
pointed out or highlighted inside the displayed full-text, in order to show the
user the places where it appears.

The solution proposed by Multivio responds to an increasing necessity following
the current boom of repositories of online documentation, stimulated among
other things by the
\href{http://en.wikipedia.org/w/index.php?title=Open_access_(publishing)&oldid=271257177}
{Open Access movement}, by the \href{http://www.openarchives.org/}{OAI-PMH}
interoperability and by the growing digitization efforts of the libraries. In
this way it concerns existing document servers such as \href{http://doc.rero.ch/}
{RERO DOC} and alike, including long-term archiving repositories allowing public
access to their contents.

\section{Goals}

\subsection{Generic}

Multivio allows the user to access different kinds of media, not only textual
but also images, audio, video and others, supporting several file types.
Emphasis is put particularly on those adopted by the Convention on Electronic
Theses by the Conference of Swiss University Libraries (CBU)
(\href{http://www.kub-cbu.ch/navi.cfm?st1=400&st2=200&st3=&st4=&w=1600&status=2}
{up-to-date version of the document}).

The retrieved documents may be simply structured, such as a book, an image or
an audio stream, or otherwise contain a complex structure such as a journal, an
image collection, a multi-volume work, a dictionary or encyclopedia, etc. which
demands the possibility of a hierarchical navigation by chapters, sections,
years, months, editions, volumes, articles, headings and other kinds of
partitioning depending on the nature of the document.

\section{Functional and complete}

The user interface is expected to be simple and intuitive yet sufficiently
complete in order to allow a nice user experience. It is composed of a set of
blocs disposed in the screen each providing a group of functionalities,
particularly: main content display area; navigation tree for structured
documents (year, month, volume, chapter, section, page...); table of contents;
page or image thumbnails; navigation commands (next, previous, first, last,
play, stop, forwards, backwards…); visualization commands (zoom, pan...).

Whenever the document is retrieved after a query, either in an external search
engine (like the document source's catalog) or through Multivio's own search
tool, it is important to highlight the queried expression inside the full-text
as well as to show the corresponding list of occurrences.

\section{Flexible}

The holders or trustees of the contents made available through Multivio should
have the possibility, if desired, of preventing users from printing or
downloading the files locally, allowing these to only be displayed inside
Multivio's own interface.

\section{Extensible and autonomous}

Multivio is an open-source project. It relies on an open and modular
architecture based on an appropriate articulation between graphical components
(representing the groups of functionalities) and connectors/plug-ins
(representing the different kinds of media). This aspect allows the tool to be
extensible and adaptable to different contexts, especially by the addition of
new features and the possibility of handling new document types and file
formats. The open character is reinforced by the use of established standards
and open protocols.

Multivio takes the form of an independent and adaptable module capable of
working together with different document servers, search engines or catalogues
using multiple and varied technologies.

\section{Sponsorship}

This project has been initiated by RERO in 2007, in partnership with the CDROM
- ''Conseil des directeurs des grandes bibliothèques de Suisse occidentale''.
In 2008 it received a 3-year funding pack in the framework of the innovation
and cooperation project \href{http://www.e-lib.ch/}{E-lib.ch: Swiss Electronic
library}, which is itself funded by the Swiss University Conference (SUK/CUS).

\end{document}
